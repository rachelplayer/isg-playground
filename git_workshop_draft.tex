\documentclass[aspectratio=169]{beamer}
\usetheme[progressbar=frametitle,block=fill]{metropolis}

% strikethoughs
\usepackage[normalem]{ulem}
\renewcommand{\ULthickness}{3pt}

% code listing syntax highlighting
\usepackage[utf8]{inputenc}
\usepackage{listings}
\usepackage{xcolor}

\definecolor{codegreen}{rgb}{0,0.6,0}
\definecolor{codegray}{rgb}{0.5,0.5,0.5}
\definecolor{codepurple}{rgb}{0.58,0,0.82}
\definecolor{backcolour}{rgb}{0.95,0.95,0.92}

\lstdefinestyle{mystyle}{
    backgroundcolor=\color{backcolour},   
    commentstyle=\color{codegreen},
    keywordstyle=\color{magenta},
    numberstyle=\tiny\color{codegray},
    stringstyle=\color{codepurple},
    basicstyle=\ttfamily\footnotesize,
    breakatwhitespace=false,         
    breaklines=true,                 
    captionpos=b,                    
    keepspaces=true,                 
    numbers=left,                    
    numbersep=5pt,                  
    showspaces=false,                
    showstringspaces=false,
    showtabs=false,                  
    tabsize=2
}

\lstset{style=mystyle}

% Style for placing info along bottom of slides
\setbeamerfont{page number in head/foot}{size=\tiny}
\setbeamercolor{footline}{fg=gray}


% back-up slides
\usepackage{appendixnumberbeamer}

% Images and figures
\usepackage{graphicx}

% Hyperlinks
\usepackage{hyperref}

\title{Github.final.finished.this-one}
\subtitle{Centre For Doctoral Training Github Session}
\author{{\large \textbf{Rachel~Player}} and {\large \textbf{Nathan~Rutherford}}\\}
\institute[RHUL]{Centre for Doctoral Training in Cyber Security\\
                 Royal Holloway, University of London}
\date[CDT'22]{May 22}

\begin{document}
    \frame{\titlepage}

    \begin{frame}[plain]
        Motiviation for using github
    \end{frame}

    \section{Setup}

        \begin{frame}
            \frametitle{Creating a Github account}
        
        \end{frame}

        \begin{frame}
            \frametitle{Authentication using SSH}
        
            
        
        \end{frame}

        \begin{frame}
            \frametitle{Visual Studio code}
        
            
        
        \end{frame}

        \begin{frame}
            \frametitle{How to use the terminal}
        
            
        
        \end{frame}

    \section{Github 101}

    \begin{frame}
        \frametitle{Git Repos}
    
        Structure
    
    \end{frame}

    \begin{frame}
        \frametitle{Creating a repo}
    
        git clone and git init
    
    \end{frame}

    \begin{frame}
        \frametitle{Add or removing files}
    
        git add and git remove
    
    \end{frame}

    \begin{frame}
        \frametitle{Saving changes}

        git commit
    
    \end{frame}

    \begin{frame}
        \frametitle{Git push}
    
        
    
    \end{frame}

    \begin{frame}
        \frametitle{Excercises}
    
        
    
    \end{frame}

    \section{Collaboration with Git}

    \begin{frame}
        \frametitle{Intro to HTML/CSS/Javascript}
    
        
    
    \end{frame}

    \begin{frame}
        \frametitle{Merge conflicts}
    
        
    
    \end{frame}

    \begin{frame}
        \frametitle{Excercise}
    
        shared html site between atendees
    
    \end{frame}

    \section{Advanced Github for creating your own personal website}

    \begin{frame}
        \frametitle{Why might you want one and why use git}
        
    
    \end{frame}

    \begin{frame}
        \frametitle{Static Sites (Jeykll)}
    
        
    
    \end{frame}

    \begin{frame}
        \frametitle{Git Fork}
    
        
    
    \end{frame}

    \begin{frame}
        \frametitle{Creating your own website}
    
        using git fork
    
    \end{frame}

    \begin{frame}
        \frametitle{Continuous Integration, Continuous Deployment (CI/CD)}
    
        Very basic what it is, and how it is used for a website
    
    \end{frame}

    \begin{frame}
        \frametitle{Editing your website details}
    
        
    
    \end{frame}

    \begin{frame}
        \frametitle{Adding a blog post}
    
        
    
    \end{frame}

    \section{Wrap-up}

    \begin{frame}
        \frametitle{Session summary}
    
        
    
    \end{frame}

    \begin{frame}
        \frametitle{Next steps and resources}
    
        
    
    \end{frame}

    
\end{document}